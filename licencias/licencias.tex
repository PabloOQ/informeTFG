\section{Licencias}
Este TFT usa códigos Open Source de diversas fuentes, así pues en esta sección se cumple el contrato puesto por estas licencias.
\subsection{Bitwarden mobile}
Bitwarden mobile  usa la licencia GPL 3.0. Como el objetivo principal de este proyecto es la modificación de este código, los cambios realizados se encuentran en el fork de GitHub de PabloOQ.
Debido a un problema de compatibilidad durante el desarrollo hubo que hacer un merge de la rama master de Bitwarden, todo el código se encuentra en:

\noindent\url{https://github.com/PabloOQ/mobile/tree/stick-after-merge}

\noindent Sin embargo, para visualizar rápidamente el código modificado antes de hacer el merge, se podría ver en esta rama (aunque este está también en la rama mencionada anteriormente):

\noindent\url{https://github.com/PabloOQ/mobile/tree/stick}

\subsection{Vaultwarden}
Vaultwarden usa la licencia GPL 3.0. Este código no ha sido modificado, simplemente se ha usado para crear una estancia vaultwarden en la que hacer pruebas de la aplicación de Bitwarden para Android.

\subsection{InputStick}
En GitHub la API de InputStick no tiene la licencia establecida, por ello se le preguntó personalmente al creador y él dió el visto bueno, sin poner ningún tipo de condición.

\subsection{GNU GPL 3}
\url{https://www.gnu.org/licenses/gpl-3.0.en.html}

% https://www.freepik.com/free-vector/blank-smartphone-icon-isolated-white-background_9688880.htm#query=phone&position=0&from_view=author
% https://www.freepik.com/free-vector/computer-design_919225.htm#query=monitor%20screen&position=1&from_view=search&track=ais
% https://www.freepik.com/free-vector/two-black-server-racks-realistic-illustration_3907757.htm