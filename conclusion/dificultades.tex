El mayor problema ha sido sin duda, entender el código de Bitwarden, sobre todo el patrón \gls{mvvm}, de hecho es algo normal en desarrolladores con no mucha experiencia\cite{García2023intro}, esto ha llevado mucho más tiempo del esperado, así como un tira y afloja constante con el código pues se ha tenido que refactorizar más veces de las necesarias por falta de comprensión en su funcionamiento. Antes de comenzar con el proyecto nuestras expectativas eran que la dificultad estuviese en usar la librería de InputStick, escrita en Java, en C\# y el uso de la librería de InputStick para la comunicación con el mismo. Sin embargo, Visual Studio se ocupa de añadir una capa intermedia para la compatibilidad con la capa, y el uso de InputStickBroadcast y InputStickUtility facilita mucho el uso de la librería. La \gls{api} de InputStick se puede usar de forma más profunda pero perdiendo las comodidades aportadas por InputStickUtility, esto era el objetivo del proyecto, sin embargo se usó la parte \quotes{simple} de la librería a modo de plantilla y a medida que el proyecto avanzaba quedaba cada vez más claro que no iba a dar tiempo de cambiarlo.