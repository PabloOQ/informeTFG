\subsubsection{Usuario de Bitwarden}
Se ha usado como usuario de prueba al mismo que originó la idea del proyecto. La opinión de dicho usuario es que es un poco complicado configurarlo, ya que no hay ningún aviso cuando la app InputStickUtility no está instalada, esto se suponía que debía ocurrir, si al intentar comunicarse con InputStickUtility esta no se encuentra instalada, \href{https://github.com/inputstick/InputStickAPI-Android/blob/81d9ce96aa9e4db4f508090f54bea981ffecfcb7/InputStickAPI/src/com/inputstick/api/broadcast/InputStickBroadcast.java\#L73}{la propia librería genera una ventana emergente}, sin embargo cuando dicho usuario lo probó, el diálogo no apareció, nuestra teoría es que está relacionado con \gls{jbl}, por lo que es algo a solucionar. El usuario plantea un tutorial de ventanas emergentes, lamentablemente esto es algo que estaría fuera de lugar en la app de Bitwarden, pues sería la única instancia donde encontrar dicho tutorial. El otro inconveniente que encontró era el hecho de tener en cuenta la distribución del teclado, pues es algo que realmente el usuario promedio no sabe que funciona así, y tampoco es fácil de comprender, lamentablemente es inevitable el tener que configurar la opción del teclado, y recae en el usuario, la única forma de solucionar esto sería explicándolo mejor. Creemos que actualmente este sistema innecesariamente complejo, y que su uso a día de hoy se debe simplemente a que es algo que se lleva usando mucho tiempo y cambiar a algo mejor sería una tarea en la que las compañías y grupos más grandes se tendrían que poner de acuerdo. Además de esto el usuario indicó que no se fijó en los nuevos botones porque asumía que la opción estaría en el menú desplegable, esto tiene fácil solución, indicándolo en la descripción de la configuración.

\subsubsection{Usuario de InputStick}
Lamentablemente, no es fácil encontrar a un usuario que conozca InputStick de antemano, así que se ha seleccionado un usuario y se le ha explicado el funcionamiento de InputStick. Tras ello y simplemente diciéndole el objetivo de la app de Bitwarden se le ha dejado navegar por la app modificada a ver si es capaz de desenvolverse. El único problema que encontró era activar la nueva funcionalidad, pues no sabía dónde descartó la configuración como pantalla para activar la funcionalidad, así que hemos intervenido, pero de resto no ha tenido mayor problema, con los textos en la pantalla de configuración lo ha comprendido fácilmente.