\usepackage[utf8]{inputenc}

\pagenumbering{arabic}

\usepackage[hyphens]{url}

\usepackage[spanish,es-tabla]{babel}
\usepackage[colorlinks, bookmarksopen, linkcolor=red ,citecolor=blue, urlcolor=blue]{hyperref}

% bonito
\usepackage{fancyhdr}
\pagestyle{fancy}

% para pruebas
\usepackage{lipsum}
\usepackage{blindtext}

\usepackage{graphicx}
\graphicspath{ {./gfx/} }

% despues de hyperref para que que funcione correctamente
\usepackage[acronym]{glossaries}
\makeglossaries
\renewcommand*{\glstextformat}[1]{\textcolor{black}{#1}}

\usepackage{float}

% para tener profundidad 4 en las secciones y en la tabla de contenidos
\setcounter{secnumdepth}{4}
\setcounter{tocdepth}{4}
%\chapter - 0
%\section - 1
%\subsection - 2
%\subsubsection - 3
%\paragraph - 4
%\subparagraph - 5

\usepackage{subfigure}

\usepackage{multicol}
\setlength{\columnsep}{1cm}
\usepackage{multirow}

\usepackage{tabularx}

\usepackage{listings}
\usepackage{xcolor}
\definecolor{codegreen}{rgb}{0,0.6,0}
\definecolor{codegray}{rgb}{0.5,0.5,0.5}
\definecolor{codepurple}{rgb}{0.58,0,0.82}
\definecolor{backcolour}{rgb}{0.95,0.95,0.92}
\lstnewenvironment{csharp}[2][]{%
  \lstset{%
    title = #2,
    #1,
    language=[Sharp]C,
    backgroundcolor=\color{backcolour},   
    commentstyle=\color{codegreen},
    keywordstyle=\color{magenta},
    numberstyle=\tiny\color{codegray},
    stringstyle=\color{codepurple},
    basicstyle=\ttfamily\footnotesize,
    breakatwhitespace=false,         
    breaklines=true,                 
    captionpos=b,                    
    keepspaces=true,                 
    numbers=left,                    
    numbersep=5pt,                  
    showspaces=false,                
    showstringspaces=false,
    showtabs=false,                  
    tabsize=2
  }%
}{}

\usepackage{pdfpages}

\usepackage{verbatim}

\usepackage{datetime}

% comandos
\newcommand\tab[1][1cm]{\hspace*{#1}}
\newcommand{\quotes}[1]{``#1''}