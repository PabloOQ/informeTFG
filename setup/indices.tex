% glosario
\newglossaryentry{vault}{name={bóveda},description={\textit{vault}. Donde Bitwarden almacena los credenciales de las distintas cuentas del usuario. Adicionalmente en este informe se usa indistintamente para referirse también a una cuenta de Bitwarden}}
\newglossaryentry{cuenta}{name=cuenta,description={Cuando se hable de cuenta se habla de a lo que un \textit{\gls{login}} hace referencia, que principalmente suelen ser cuentas de servicios web. Cuando se refiera a una cuenta de Bitwarden se usará el término \gls{vault}}}
\newglossaryentry{login}{name={elemento de cuenta},description={\textit{Login item}: Par de credenciales Usuario-Clave\cite{item}}}
\newglossaryentry{freemium}{name=\textit{freemium},description={De \textit{free} y \textit{premium}. Que ofrece un servicio gratuito además de una opción de pago}}

% acronimos y siglas
% si provienen del inglés están en cursiva
\newacronym{usb}{USB}{\textit{Universal Serial Bus}}
\newacronym{hid}{HID}{\textit{Human Interface Device}}
\newacronym{api}{API}{\textit{Application Programming Interface}}
\newacronym{vpn}{VPN}{\textit{Virtual Private Network}}
\newacronym{cli}{CLI}{\textit{Command Line Interface}}
\newacronym{aes}{AES}{\textit{Advanced Encryption Standard}}
\newacronym{sha}{SHA}{\textit{Secure Hash Algorithm}}
\newacronym{e2ee}{E2EE}{\textit{End-to-end encryption}}
\newacronym{url}{URL}{\textit{Uniform Resource Locator}}
\newacronym{uri}{URI}{\textit{Uniform Resource Identifier}}
\newacronym{totp}{TOTP}{\textit{Time-based One-time Password}}
\newacronym{2fa}{2FA}{\textit{Two-factor Authentication}}
\newacronym{hmac}{HMAC}{\textit{Hash-based Message Authenthication Code}}
\newacronym{pnp}{PnP}{\textit{Plug and Play}}
\newacronym{ble}{BLE}{\textit{Bluetooth Low Energy}}
\newacronym{bios}{BIOS}{\textit{Basic Input/Output System}}
\newacronym{pbkdf}{PBKDF}{\textit{Password-Based Key Derivation Function}}
\newacronym{srp}{SRP}{\textit{Secure Remote Password}}
\newacronym{qmk}{QMK}{\textit{Quantum Mechanical Keyboard}}
\newacronym{zmk}{ZMK}{\textit{Zephyr Mechanical Keyboard}}
\newacronym{mvvm}{MVVM}{\textit{Model-View-ViewModel}}
\newacronym{mvc}{MVC}{\textit{Model-View-Controller}}
\newacronym{mvp}{MVP}{\textit{Model-View-Presenter}}
\newacronym{jbl}{JBL}{\textit{Java Bindings Library}}
\newacronym{uml}{UML}{\textit{Unified Modeling Language}}
\newacronym{gui}{GUI}{\textit{Graphical User Interface}}
\newacronym{cryptomac}{MAC}{\textit{Message Authentication Code}}
%\newacronym{addressmac}{MAC}{\textit{Media Access Control}}

%-----------------------

{
    \hypersetup{linkcolor=black}
    \tableofcontents
    \newpage

    % figuras
    \listoffigures
    \newpage

    % tablas
    \listoftables
    \newpage
}
% glosario
\printglossary
\newpage

% acrónimos
\setacronymstyle{long-short}
\printglossary[title={Siglas y acrónimos},type=\acronymtype,nonumberlist]
%,title=Abreviaciones, toctitle=Índice de abreviaciones